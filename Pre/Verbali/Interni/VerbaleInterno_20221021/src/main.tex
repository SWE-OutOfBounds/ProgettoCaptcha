\documentclass[12pt]{article}

% Impostazioni generali

\usepackage[utf8]{inputenc}
\usepackage[italian]{babel}
\selectlanguage{italian}
\usepackage{tabularx}
\usepackage{multirow}
\usepackage{geometry}
\geometry{
    a4paper,    
    left=20mm,  
    right=20mm, 
    top=30mm, 
    bottom=20mm,
    headheight=50pt
    }
\usepackage{eso-pic,graphicx, transparent}
\usepackage{graphicx}
\usepackage{hyperref}

\hypersetup{
    colorlinks=true,
    linkcolor=black,
    urlcolor=black,
    pdftitle={Verbale riunione 21/10/22 v 1.0} %AGGIUNGI DATA E VERSIONE
   }

% Header e footer delle pagine

\usepackage{fancyhdr, lastpage}
\pagestyle{fancy}
\lhead{\includegraphics[width=0.3in,height=0.3in]{logov2.png} \large \bfseries Out of Bounds}
\rhead{\textsl{Verbale riunione 21/10/22 v 1.0}}  
\rfoot{Pagina \thepage  \thinspace\thinspace di \pageref{LastPage}}
\lfoot{\href{mailto::sweoutofbounds@gmail.com}{sweoutofbounds@gmail.com}}
\cfoot{}
\renewcommand{\footrulewidth}{0.4pt}

%%%%%%%%%%%%%%%%%%%%%%%%%%%%%%%%%%%%%%%%%%%%%%%%%%%%%%%%%%%%%%%%%%%%%%%%%%%%%%%%%%%%%%%%%%%%%%%


\begin{document}


\thispagestyle{empty}
\setcounter{page}{0}

\begin{center}



%\AddToShipoutPictureBG*{\transparent{0.2} %\includegraphics[width=\paperwidth,height=\paperheight]{logov2.png}};


\includegraphics[width=1.5in,height=1.5in]{logov2.png}\\
\Large
\textsc{Out of Bounds}\\

\vspace{1cm}
\textsc{Verbale di Riunione}\\
versione 1.0 data 21/10/22\\

\vspace{4cm}

\begin{tabular}{r|l}
    \textbf{Uso}            &  Interno\\
    \textbf{Redazione}      &  Michele Cazzaro\\
    \textbf{Verifica}       &  Jacopo Angeli\\
                            &  Valentina Caputo\\
    \textbf{Destinatari}    &  Out of Bounds\\
    \textbf{}               &  Prof. Tullio Vardanega\\
    \textbf{}               &  Prof. Riccardo Cardin\\
    
\end{tabular}

\vfill

%% footer

\normalsize
\rule{8cm}{0.1mm}\\
\bigskip
contatti Out of Bounds\\
sweoutofbounds@gmail.com
\end{center}
\clearpage


\tableofcontents
\clearpage

\section{Informazioni generali} 
\begin{itemize}
    \item Luogo: Stanza virtuale Zoom
    \item Data: 21/10/22
    \item Orario di inizio: 12.30
    \item Orario di fine: 13:30
    \item Lista partecipanti:
\end{itemize}
\begin{center}
\begin{tabularx}{0.8\textwidth} { 
   >{\raggedright\arraybackslash}X 
   >{\centering\arraybackslash}X 
   >{\raggedleft\arraybackslash}X  }
\hline
    \textbf{Nome e Cognome} &        \textbf{presente}\\
\hline
    Jacopo Angeli           &        \emph{presente}\\
    Simone Bisortole        &        \emph{presente}\\
    Valentina Caputo        &        \emph{presente}\\
    Michele Cazzaro         &        \emph{presente}\\
    Alberto Matterazzo       &        \emph{presente}\\
    Edoardo Retis           &        \emph{presente}\\
\end{tabularx}
\end{center}

\vspace{5mm}

\section{Ordine del giorno}
\begin{enumerate}
    \item   Conoscenza dei membri del gruppo
    \item   Brand identity del gruppo
    \item   Decisione di mezzi di comunicazione comuni
    \item   Discussione dei capitolati per la scelta
    \item   Decisione della cadenza degli incontri tra membri
    \item   Creazione e-mail di gruppo
\end{enumerate}

\vspace{5mm}

\section{Resoconto}
Dopo esserci brevemente presentati, il gruppo ha confermato la già presente volontà di utilizzare \textbf{Telegram} come piattaforma principale per le conversazioni informali e veloci.

Il nome ed il logo del gruppo sono stati confermati in seguito ad uno scambio di messaggi effettuato su Telegram.

La scelta dello strumento per le riunioni invece è ricaduta senza discussioni particolari su \textbf{Discord}.

Per la condivisione di documenti e codice sorgente il gruppo decide che utilizzerà \textbf{Github}.

Durante la discussione per la scelta dei capitolati, è stata formalizzata la preferenza stretta per 3 delle proposte, come previsto dal sondaggio effettuato su Telegram. Il capitolato preferito risulta essere C1, proposto da Zucchetti; è seguito rispettivamente da C5 e C4.

Il giorno in cui è stata effettuata la prima riunione sembra adatto per delle riunioni ricorrenti (Venerdì mattina). Si rende evidente la necessità di effettuare almeno un incontro collettivo con cadenza settimanale.

In seguito a dei problemi riscontrati con Google Groups, il gruppo decide di utilizzare \textbf{Gmail} per l'indirizzo mail del gruppo. Si decide di implementare l'inoltro automatico alle caselle universitarie dei membri.

Viene esternata la necessità di stilare un documento che formalizzi il way of working.

La riunione si scioglie con l'intenzione di incontrarsi personalmente dopo lezione all'inizio della settimana ventura.

\vspace{5mm}

\section{Note finali}
\begin{itemize}
    \item Comunicazione testuale: Telegram
    \item Comunicazione verbale: Discord
    \item Comunicazioni formali: Gmail
    \item Condivisione materiale: Github
    \item Effettuata scelta del logo
    \item Scelta della preferenza dei capitolati    \begin{enumerate}
                                                        \item C1
                                                        \item C5
                                                        \item C4
                                                    \end{enumerate}
    \item Scelta cadenza riunioni
\end{itemize}
\end{document}
