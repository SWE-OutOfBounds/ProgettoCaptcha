\documentclass[12pt]{article}

% PACKAGE LIST
\usepackage[utf8]{inputenc}
\usepackage{geometry}
\usepackage{tabularx}
\usepackage{multirow}%package to manage images
\usepackage[italian]{babel}
\usepackage{hyperref}
\usepackage{fancyhdr, lastpage}
\usepackage{eso-pic,graphicx, transparent}
\usepackage{ifthen}

%% CONFIGURATION 
\graphicspath{ {./images/} }
%DocumentInfo
\def \docTitle {Titolo del Documento}
\def \pdfTitle {Titolo del Pdf}
\def \documentVersion {} %Add newLine after version (E.G. "Versione 0.0.1 \\")
\def \firstEditor {Primo Editore}
\def \secondEditor {\hfill}
\def \thirdEditor {\hfill}
\def \verifier {Verificatore}

%TeamInfo
\def \teemMail {sweoutofbounds@gmail.com}
\def \documentsRepo {https://github.com/SWE-OutOfBounds/Documents}
\def \softwareRepo {}

% Attenzione : aggiungere \listoffigures \listoftables se necessario 
% (list of tables tiene conto delle tabelle definite usando \begin{table} (macro di latex ez))

%Document Body
\def \documentBody{
    %Scrivere qui il corpo del documento

}

\newcolumntype{R}[1]{>{\raggedleft\let\newline\\\arraybackslash\hspace{0pt}}m{#1}}
\newcommand*{\thead}[1]{\multicolumn{1}{c}{\bfseries #1}}
\renewcommand{\arraystretch}{1.3}%

\geometry{
    a4paper,    
    left=20mm,  
    right=20mm, 
    top=30mm, 
    bottom=20mm,
    headheight=50pt
    }
\hypersetup{
    colorlinks=true,
    linkcolor=black,
    urlcolor=black,
    pdftitle={\pdfTitle} %AGGIUNGI DATA E VERSIONE
   }

\pagestyle{fancy}
\fancyhead{}
\fancyhead[L]{\includegraphics[width=0.3in,height=0.3in]{OutOfBoundsLogo.png} \large \bfseries Out of Bounds}
\fancyhead[R]{\textsl{\pdfTitle}}
% Footer delle pagine
\fancyfoot{}
\fancyfoot[L]{\href{mailto:\teemMail}{\teemMail}}
\fancyfoot[R]{Pagina \thepage\ di \pageref{LastPage}}
\rfoot{Pagina \thepage\ di \pageref{LastPage}}
\renewcommand{\footrulewidth}{0.4pt}

%% DOCUMENT BODY

\begin{document}

%% FIRST PAGE

\thispagestyle{empty}
\setcounter{page}{0}

\begin{center}
\includegraphics[width=1.5in,height=1.5in]{OutOfBoundsLogo.png}\\
\Large
\textsc{Out of Bounds}\\
\vspace{2cm}
\Huge
\textsc{Scelta del capitolato}\\
\Large
\documentVersion
\vspace{3cm}

\Large
\begin{tabular}{R{8cm}|p{8cm}}
    \textbf{Redazione}      &  \firstEditor\\
    % \textbf{Redazione}      &  \secondEditor\\ 
    % \textbf{Redazione}      &  \thirdEditor\\ 
    \textbf{Verifica}       &  \verifier\\
    \textbf{Uso}            &  Interno\\
    \textbf{Destinatari}    &  Out of Bounds\\
    \hfill                  &  prof. Tullio Vardanega\\
    \hfill                  &  Prof. Riccardo Cardin\\
\end{tabular}

\vfill

\normalsize
\rule{8cm}{0.1mm}\\
\bigskip
\textsc{Contatti}\\
\href{mailto:\teemMail}{\teemMail}\\
\textsc{Repositories}\\
\href{\documentsRepo}{Documentale}\\
\href{\softwareRepo}{Software} 
\end{center}

%% ALTRE PAGINE
\newpage

\tableofcontents

\newpage

\section{Motivazione della scelta}
Dopo aver esaminato in dettaglio le proposte di progetto delle varie aziende, il gruppo ha deciso di scegliere il capitolato C1 per le seguenti ragioni:
\begin{itemize}
    \item Il gruppo crede fortemente nella futura applicazione globale della tecnologia CAPTCHA ed è stato votato dal gruppo come prima scelta all'unanimità,
    \item L’azienda si è da subito resa molto disponibile, per quanto riguarda il materiale per la formazione e la libera scelta sulle tecnologie da utilizzare,
    \item Il team di progetto è rimasto entusiasta dall'interesse mostrato dal proponente alle proprie proposte di innovazione del capitolato,
    \item La presentazione del capitolato è stata chiara e completa,
    \item L’azienda proponente possiede una buona esperienza riguardante il seguente progetto, quindi il proponente è stato reputato in grado di seguire adeguatamente il team di sviluppo durante tutta la durata del percorso.
\end{itemize}

\section{Resoconto verbale esplorativo}
In data 25/10/2022 il gruppo ha avuto un incontro conoscitivo con il rappresentante dell'azienda
proponente del capitolato C1, Dott. Gregorio Piccoli, per approfondire vari punti del capitolato,
in particolare:
\begin{itemize}
    \item Problemi riguardanti accessibilità della tecnologia CAPTCHA,
    \item Problemi relativi alla sicurezza per l’accesso alla login protetta dal CAPTCHA,
    \item Dubbi relativi a sessioni con JavaScript disabilitato,
    \item Dubbi relativi a sessioni con JavaScript disabilitato,
\end{itemize}
In conclusione, il gruppo è stato soddisfatto della riunione e il capitolato è sembrato molto
interessante. Per approfondire i temi trattati durante l'incontro conoscitivo, si consiglia la visione
del verbale di riferimento.

\end{document}
